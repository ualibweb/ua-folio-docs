\documentclass{article}

\usepackage[paperwidth=\maxdimen,paperheight=\maxdimen]{geometry}

\usepackage{tikz}
\usetikzlibrary{positioning, calc}

\usepackage[tightpage,active]{preview}
\PreviewEnvironment{tikzpicture}

\begin{document}

\begin{tikzpicture}
  [
    ->,
    >=stealth,
    node distance=1cm and 4cm,
    every text node part/.style={align=center}
  ]
  % Original mod-calendar permissions
  \node[rectangle, draw, fill=blue!30] (calendar-opening-hours-collection-get) {\texttt{calendar.opening-hours.collection.get} \\ List calendar opening hours};
  \node[rectangle, draw, fill=blue!30, below=of calendar-opening-hours-collection-get] (calendar-periods-collection-get) {\texttt{calendar.periods.collection.get} \\ List calendar periods descriptions};
  \node[rectangle, draw, fill=blue!30, below=of calendar-periods-collection-get] (calendar-periods-item-get) {\texttt{calendar.periods.item.get} \\ Get period description};
  \node[rectangle, draw, fill=blue!30, below=of calendar-periods-item-get] (calendar-periods-item-post) {\texttt{calendar.periods.item.post} \\ Add new calendar period description};
  \node[rectangle, draw, fill=blue!30, below=of calendar-periods-item-post] (calendar-periods-item-put) {\texttt{calendar.periods.item.put} \\ Update existing calendar period description};
  \node[rectangle, draw, fill=blue!30, below=of calendar-periods-item-put] (calendar-periods-item-delete) {\texttt{calendar.periods.item.delete} \\ Remove calendar period description};

  % New mod-calendar permissions
  \node[rectangle, draw, double=green, double distance=1mm, left=4cm of calendar-periods-collection-get] (calendar-view) {\texttt{calendar.view} \\ Access and query calendar information};
  \node[rectangle, draw, double=green, double distance=1mm, below left=1.5cm and 1.5cm of calendar-view] (calendar-create) {\texttt{calendar.create} \\ Create new calendars};
  \node[rectangle, draw, double=green, double distance=1mm, below=of calendar-create] (calendar-update) {\texttt{calendar.update} \\ Edit existing calendars};
  \node[rectangle, draw, double=green, double distance=1mm, below=of calendar-update] (calendar-delete) {\texttt{calendar.delete} \\ Delete calendars};

  % New ui-calendar permissions
  \node[rectangle, draw, left=6cm of calendar-view, fill=green!30] (ui-calendar-view) {\texttt{ui-calendar.view} \\ Can view calendar configuration};
  \node[rectangle, draw, below left=0.5cm and 2cm of ui-calendar-view, fill=green!30] (ui-calendar-create) {\texttt{ui-calendar.create} \\ Can create new calendars};
  \node[rectangle, draw, double=green, double distance=1mm, below=0.5cm of ui-calendar-create, fill=green!30] (ui-calendar-update) {\texttt{ui-calendar.update} \\ Can edit existing calendars};
  \node[rectangle, draw, double=green, double distance=1mm, below=0.5cm of ui-calendar-update, fill=green!30] (ui-calendar-delete) {\texttt{ui-calendar.delete} \\ Can delete calendars};

  % Old ui-calendar permissions
  \node[rectangle, draw, left=1cm of ui-calendar-update, fill=green!30] (ui-calendar-edit) {\texttt{ui-calendar.edit} \\ Can edit calendar events (no removal)};
  \node[rectangle, draw, below left=0.5cm and 1cm of ui-calendar-edit, fill=green!30] (ui-calendar-all) {\texttt{ui-calendar.all} \\ Can create, view, edit, and remove calendar events};
  % Removed
  \draw[-][red, very thick, line cap=round] ($(ui-calendar-edit.north west)+(.02,-.02)$) -- ($(ui-calendar-edit.south east)+(-.02,.02)$);
  % Removed
  \draw[-][red, very thick, line cap=round] ($(ui-calendar-all.north west)+(.02,-.02)$) -- ($(ui-calendar-all.south east)+(-.02,.02)$);

  % UI Border
  \draw[red,very thick,dotted] ($(ui-calendar-all.south west)+(-0.6,-0.2)$)  rectangle ($(ui-calendar-view.north east)+(0.2,0.2)$);

  % Original calendar.all
  \node[rectangle, draw, below=of ui-calendar-delete, node distance=4cm] (calendar-all) {\texttt{calendar.all} \\ All permission for mod-calendar module};
  % Removed
  \draw[-][red, very thick, line cap=round] ($(calendar-all.north west)+(.02,-.02)$) -- ($(calendar-all.south east)+(-.02,.02)$);

  % New mod-calendar to old mod-calendar
  \draw [->] (calendar-view) to [out=0, in=180] (calendar-opening-hours-collection-get);
  \draw [->] (calendar-view) to [out=0, in=180] (calendar-periods-collection-get);
  \draw [->] (calendar-view) to [out=0, in=180] (calendar-periods-item-get);
  \draw [->] (calendar-create) to [out=0, in=180] (calendar-periods-item-post);
  \draw [->] (calendar-update) to [out=0, in=180] (calendar-periods-item-put);
  \draw [->] (calendar-delete) to [out=0, in=180] (calendar-periods-item-delete);
  \draw [->] (calendar-create) to [out=0, in=180] (calendar-view);
  \draw [->] (calendar-update) to [out=0, in=180] (calendar-view);
  \draw [->] (calendar-delete) to [out=0, in=180] (calendar-view);
  % New ui-calendar to new mod-calendar
  \draw [->] (ui-calendar-view) to [out=0, in=180] (calendar-view);
  \draw [->] (ui-calendar-create) to [out=0, in=180] (calendar-create);
  \draw [->] (ui-calendar-update) to [out=0, in=180] (calendar-update);
  \draw [->] (ui-calendar-delete) to [out=0, in=180] (calendar-delete);
  % New UI to new UI
  \draw [->] (ui-calendar-create) to [out=0, in=180] (ui-calendar-view);
  \draw [->] (ui-calendar-update) to [out=0, in=180] (ui-calendar-view);
  \draw [->] (ui-calendar-delete) to [out=0, in=180] (ui-calendar-view);
  % Old UI to new UI
  \draw [->, red] (ui-calendar-all) to [out=0, in=180] (ui-calendar-edit);
  \draw [->, red] (ui-calendar-all) to [out=0, in=180] (ui-calendar-delete);
  \draw [->, red] (ui-calendar-edit) to [out=0, in=180] (ui-calendar-update);
  \draw [->, red] (ui-calendar-edit) to [out=0, in=180] (ui-calendar-create);

  % Old calendar.all arrows to old permissions
  \coordinate[left=1.8cm of calendar-periods-item-delete] (routing) {};
  \draw [->, red] (calendar-all) |- (routing) |- (calendar-opening-hours-collection-get);
  \draw [->, red] (calendar-all) |- (routing) |- (calendar-periods-collection-get);
  \draw [->, red] (calendar-all) |- (routing) |- (calendar-periods-item-get);
  \draw [->, red] (calendar-all) |- (routing) |- (calendar-periods-item-post);
  \draw [->, red] (calendar-all) |- (routing) |- (calendar-periods-item-put);
  \draw [->, red] (calendar-all) |- (routing) |- (calendar-periods-item-delete);

  % Legend
  \matrix [draw,fill=white,left=-4cm of current bounding box.north west] {
    \node [shape=rectangle, draw=black,    fill=white, line width=1, label=right:Permission] {}; \\
    \node [shape=rectangle, draw=black, fill=green!30, line width=1, label=right:Assignable Permission] {}; \\
    \node [shape=rectangle, draw=black,  fill=blue!30, line width=1, label=right:Checked Permission] {}; \\
    \node [shape=rectangle, draw=black, double=green, double distance=0.6mm,  fill=white, line width=1, label=right:New Permission] {}; \\
    \node [shape=rectangle,   draw=red,        dotted,   very thick, label=right:UI Permissions] {}; \\
    \node [shape=rectangle, draw=white,    fill=white, line width=1, label=right:Sub Permission] (subperm) {}; \\
    \node [shape=rectangle, draw=white,    fill=white, line width=1, label=right:Removed Sub Permission] (subpermremoved) {}; \\
    \node [shape=rectangle, draw=black,    fill=white, line width=1, label=right:Removed Permissions] (removed) {}; \\
  };
  \draw[->] ($(subperm.north west)+(.02,-.02)$) -- ($(subperm.south east)+(-.02,.02)$);
  \draw[->, red] ($(subpermremoved.north west)+(.02,-.02)$) -- ($(subpermremoved.south east)+(-.02,.02)$);
  \draw[-][red, very thick, line cap=round] ($(removed.north west)+(.02,-.02)$) -- ($(removed.south east)+(-.02,.02)$);
\end{tikzpicture}

\end{document}
